\documentclass[11pt]{article}
\usepackage{amsmath}
\usepackage{amsfonts}
\usepackage{amsthm}
\usepackage[utf8]{inputenc}
\usepackage[margin=0.75in]{geometry}

\title{CSC111 Winter 2024 Project 1}
\author{Rachel Deng, Jeha Park}
\date{\today}

\begin{document}
\maketitle

\section*{Enhancements}


\begin{enumerate}

\item Furniture/LockedFurniture in world
	\begin{itemize}
	\item This enhancement acts like a furniture object at a location.
	Just like real life, items can be stored inside Furniture objects.
    Items inside furniture objects can only be picked up and interacted with if the furniture has been opened.
    Players can receive points for opening Furniture by typing the action ``open \{furniture name\}''.
    The LockedFurniture class inherits the Furniture class.
    When a player opens an instance of LockedFurniture, they are prompted to enter key.
    An instance of a Furniture can have special custom actions specified in the items.txt file.
    For example, the blackboard in the game can be read by the action ``read blackboard''.
	\item High
	\item We believe this enhancement is of high complexity because it made it a lot more complex to read the items.txt file
        Furniture objects and Items that are stored in furniture need to be formatted differently in the text file, which required code to check if a line
        indicated that an ``item'' in the text file was actually a furniture object.
        Then, there would be a different case to initializing the Furniture object.
        Additionally, code had to be written that did not allow a player to pick up an item if it was stored in an unopened Furniture.
        Consequently, Furniture had to have attributes that represented opened status and Items in furniture
        LockedFurniture also inherits the Furniture class, but the method, open, had to be overridden because it has to read
        input from the user and check if it equals the key stored in the key instance attribute to be opened.
        Furthermore, items stored in Furniture that were picked up after opening a Furniture object could be dropped anywhere
        and then picked up in that location without being ``stored in furniture.''
        So, the drop method for the Item class had to be implemented such that an Item no longer had a corresponding Furniture
        attribute when the method was called.
        This enhancement required a lot of consideration of cases for implementation.
	\end{itemize}

\item Custom actions for Item and Furniture objects
	\begin{itemize}
	\item Basic description of what the enhancement is:
	\item Low
	\item Reasons you believe this is the complexity level (e.g. mention implementation details
	% Feel free to add more subheadings if you feel the need
        had to include these actions and which items/furniture objects it could be performed on
        when a player called menu
	\end{itemize}

% Uncomment below section if you have more enhancements; copy-paste as many times as needed
%\item Describe your enhancement here
%	\begin{itemize}
%	\item Basic description of what the enhancement is:
%	\item Complexity level (low/medium/high):
%	\item Reasons you believe this is the complexity level (e.g. mention implementation details
%	% Feel free to add more subheadings if you feel the need
%	\end{itemize}

\end{enumerate}


\section*{Extra Gameplay Files}

If you have any extra \texttt{gameplay\#.txt} files, describe them below.

\end{document}
